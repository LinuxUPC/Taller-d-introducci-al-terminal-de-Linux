\section{Introducció al taller}

\subsection{Sobre el curs}

\par
En aquest curs es dóna una introducció global del funcionament del sistema operatiu Linux.
Aquest sistema operatiu, com molts sabeu, està cobrant molta importància, no tan sols a les universitats i centres d'investigació, sinó també en empreses.
Això es deu al fet que és un sistema molt estable, permet ``multitasca'' i multiusuari.
\par
Al llarg d'aquest taller ensenyarem a utilitzar aquest sistema, a través del terminal.

\subsection{A qui va dirigit}

\par
La informació del que tractarem en aquest curs pot ser trobada a Internet, fins i tot manuals per donar els primers passos a Linux.
És molt útil tenir ajuda, per seguir aquests passos, així com, tenir exemples i exercicis que ajudaran a assimilar els conceptes, per tant aquest és un taller per aquelles persones que comencen des de zero a utilitzar aquest sistema i volen rebre un cop de mà per començar aprendre més ràpid.

\subsection{Entorn}

\par
Aquest taller ha estat preparat amb i per usuaris del sistema operatiu GNU/Linux intentant
ser agnostic de la distribució d'aquest mateix. En concret, cobrirem les distribucions basades
en Debian\footnote{\url{www.debian.org}} i Arch Linux\footnote{\url{www.archlinux.org}}. 
De totes maneres la majoria de conceptes d'aquest treball són aplicables a la majoria de
distribucions GNU/Liunx.

\subsection{Nota per a traductors}

\par
Si voleu traduïr aquest treball, podeu adreçar-vos a nosaltres a \href{mailto:linuxupc@linuxupc.upc.edu}{linuxupc@linuxupc.upc.edu}
on podem afegir la traducció a aquest repositori o enllaçar la traducció des del repositori principal.
